%%%%%%%%%%%%%%%%%%%%%%%%%%%%%%%%%%%%%%%%%
% Simple Article
% Integrated article template with simple for make4ht
% LaTeX Class
% Version 1.0 (10/11/20)
%
% This class originates by:
% Vel and  Nicolas Diaz
%
% Authors:
% Muhammad Uliah Shafar
%
%
% Free License:
%
%
%%%%%%%%%%%%%%%%%%%%%%%%%%%%%%%%%%%%%%%%%
\documentclass[12pt, landscape]{simrep} % Font size (can be 10pt, 11pt or 12pt)
% Heading
\newcommand{\logo}{
	\includegraphics[scale=.2]{figures/logo.png}
}

\newcommand{\kode}{
	kode
}

% line 2
\newcommand{\mk}{xxx}
\newcommand{\kodemk}{xxx}
\newcommand{\rmk}{tidak tahu}
\newcommand{\smt}{1 (SATU)}
\newcommand{\sks}{2 (DUA)}
\newcommand{\datecreated}{09-10-2023}

% line 3
\newcommand{\dosenutama}{M. Uliah Shafar S.Ars.,M.Ars.,}
\newcommand{\dosenkedua}{M. Uliah Shafar S.Ars.,M.Ars.,}
\newcommand{\dosenketiga}{Imam Fadly S.T.,M.T.,}

% cpl
\newcommand{\cpla}{Lorem}
\newcommand{\cplb}{Lorem}
\newcommand{\cplc}{Lorem}
\newcommand{\cpld}{Lorem}

% cpmk
\newcommand{\cpmka}{Lorem}
\newcommand{\cpmkb}{Lorem}
\newcommand{\cpmkc}{Lorem}
\newcommand{\cpmkd}{Lorem}

% sub-cpmk
\newcommand{\scpmka}{Lorem}
\newcommand{\scpmkb}{Lorem}
\newcommand{\scpmkc}{Lorem}
\newcommand{\scpmkd}{Lorem}

\newcommand{\deskmk}{Lorem}
\newcommand{\bhnkaj}{Lorem}
\newcommand{\pstk}{Lorem}
\newcommand{\mksyrt}{Lorem}

%----------------------------------------------------------------------------------------
%	TITLE SECTION
%----------------------------------------------------------------------------------------
% MAIN TITLE SECTION
\title{
	\textbf{Lorem Ipsum Lorem Ipsum \\
		Lorem Ipsum Lorem Ipsum Lorem Ipsum} \\
	\textbf{{Lorem Ipsum Lorem Ipsum \\}}
} % Title and subtitle
%\date{\textbf{\DTMtoday}}
\date{\textbf{\today}}
\author{
	\begin{tabular}{@{}ll@{}}
		Nama  &  : Muhammad Uliah Shafar  \\
		NIM   &  : 21020119420029  \\
	\end{tabular}
}

%----------------------------------------------------------------------------------------
% OTHER TITLE SECTION

%\title{\textbf{Sistem Sarana dan Prasarana Jl. Pinggir Laut} \\ {\Large\itshape Infrastructure of Waterfront Parepare City}} % Title and subtitle

%\author{\textbf{Uliah Shafar} \\ \textit{Universitas Diponegoro}} % Author and institution

%\date{\today} % Date, use \date{} for no date

%----------------------------------------------------------------------------------------



\begin{document}
%----------------------------------------------------------------------------------------
%	ESSAY BODY
%----------------------------------------------------------------------------------------
\begin{comment}

\setlength\LTleft{0pt}
\setlength\LTright{0pt}
% \begin{longtable}{|*{8}{l|}}
\begin{longtable}{*{8}{P{1.15in}}@{}}
	% \caption{Lorem} \\
	% \label{tab:lorem} \\
	\toprule
	\logo                  & \multicolumn{5}{c}{ \large\thead{UNIVERSITAS MUHAMMADIYAH PAREPARE                                                                                                                                    \\  FAKULTAS TEKNIK\\ JURUSAN PERANCANGAN WILAYAH KOTA}} & \multicolumn{2}{c}{\kode} \\
    \bottomrule
	\multicolumn{8}{c}{\bfseries rencana pembelajaran semester}                                                                                                                                                                                            \\
    \midrule
    \multicolumn{2}{l}{\bfseries mata kuliah (MK)}       & \textbf{kode}                                                         & \textbf{rumpun mata kuliah}                 & \textbf{bobot (sks)}                             & \textbf{semester} & \multicolumn{2}{l}{\bfseries tanggal penyusunan} \\
    \midrule
	\multicolumn{2}{l}{\mk}                    & \kodemk                                                      & \rmk                               & \sks                                    & \smt     & \multicolumn{2}{l}{\datecreated}       \\
    \midrule
	\multicolumn{2}{l}{\bfseries otorisasi / pengesahan} & \multicolumn{2}{l}{\bfseries dosen pengembang rps}                     & \multicolumn{2}{l}{\bfseries Kordinator rmk} & \multicolumn{2}{l}{\bfseries ketua program studi}                                                     \\
    \midrule
	\multicolumn{2}{l}{}                       & \multicolumn{2}{l}{\dosenutama}                              & \multicolumn{2}{l}{\dosenkedua}    & \multicolumn{2}{l}{\dosenketiga}                                                            \\
    \midrule
    \multirow{5}{*}{\makecell{capaian\\ pembelajaran}}      & \multicolumn{3}{l}{CPL-PRODI yang dibebankan pada MK}        & \multicolumn{4}{l}{}                                                                                                             \\
    \cline{2-8}

                                          & CPL-1                                                        & \multicolumn{6}{l}{\cpla}                                                                                                        \\

    \cline{2-8}

                                           & CPL-2                                                        & \multicolumn{6}{l}{\cplb}                                                                                                        \\

    \cline{2-8}
	                                           & CPL-3                                                        & \multicolumn{6}{l}{\cplc}                                                                                                        \\

    \cline{2-8}
	                                           & CPL-4                                                        & \multicolumn{6}{l}{\cpld}                                                                                                        \\
                                          \midrule

	                                           & CPMK-1                                                       & \multicolumn{6}{l}{\cpmka}                                                                                                       \\

    \cline{2-8}

	                                           & CPMK-2                                                       & \multicolumn{6}{l}{\cpmkb}                                                                                                       \\

    \cline{2-8}
	                                           & CPMK-3                                                       & \multicolumn{6}{l}{\cpmkc}                                                                                                       \\

    \cline{2-8}
	                                           & CPMK-4                                                       & \multicolumn{6}{l}{\cpmkd}                                                                                                       \\

                                               \midrule

	                                           & Sub-CPMK-1                                                   & \multicolumn{6}{l}{\scpmka}                                                                                                      \\

    \cline{2-8}

	                                           & Sub-CPMK-2                                                   & \multicolumn{6}{l}{\scpmkb}                                                                                                      \\

    \cline{2-8}
	                                           & Sub-CPMK-3                                                   & \multicolumn{6}{l}{\scpmkc}                                                                                                      \\

    \cline{2-8}
	                                           & Sub-CPMK-4                                                   & \multicolumn{6}{l}{\scpmkd}                                                                                                      \\

                                               \midrule
	deskripsi singkat MK                       & \multicolumn{7}{l}{\deskmk}                                                                                                                                                                     \\

                                               \midrule

	bahan kajian: materi pembelajaran          & \multicolumn{7}{l}{\bhnkaj}                                                                                                                                                                     \\
                                               \midrule
	pustaka                                    & \multicolumn{7}{l}{\pstk}                                                                                                                                                                       \\
                                               \midrule

	dosen pengampu                             & \multicolumn{7}{l}{\dosenutama}                                                                                                                                                                 \\
                                               \midrule
	mata kuliah syarat                         & \multicolumn{7}{l}{\mksyrt}                                                                                                                                                                     \\
	\bottomrule
\end{longtable}


\end{comment}


\begin{longtable}{@{}l@{}P{1.25in}*{6}{P{1.15in}}@{}}
	% \caption{Lorem} \\
	% \label{tab:lorem} \\
	\toprule
    \multirow{2}{*}{mg ke-} & \multirow{2}{*}{\makecell[c]{Sub-CPMK\\ (sbg kemampuan\\ akhir diharapkan)}} & \multicolumn{2}{c}{Penilaian} & \multicolumn{2}{c}{\multirow{2}{*}{\makecell[c]{bentuk pembelajaran;\\ metode pembelajaran;\\ penugasan mahasiswa; (est. wktu)}}} & \multirow{2}{*}{\makecell[c]{materi\\ pembelajaran\\ (pustaka)}} & \multirow{2}{*}{\makecell[c]{bobot\\ penilaian\\ (\%)}} \\
                            &&indikator&kriteria \& bentuk&&&&\\
                            \bottomrule
    \multicolumn{1}{c}{(1)} &\multicolumn{1}{c}{(2)}  & \multicolumn{1}{c}{(3)} & \multicolumn{1}{c}{(4)} & \makecell[c]{tatap muka\\ (5)} & \makecell[c]{daring\\ (6)} & \multicolumn{1}{c}{(7)} & \multicolumn{1}{c}{(8)} \\
    \midrule
    1 & Sub-CPMK 1: \cpmka  & \idka & \krta & \mpa & \mpa & \mppa & \bpa\\
    2 & Sub-CPMK 2: \cpmkb  & \idkb & \krtb & \mpb & \mpb & \mppb & \bpb\\
    3 & Sub-CPMK 3: \cpmkc  & \idkc & \krtc & \mpc & \mpc & \mppc & \bpc\\
    4 & Sub-CPMK 4: \cpmkd  & \idkd & \krtd & \mpd & \mpd & \mppd & \bpd\\
    \midrule
    8 & \multicolumn{7}{c}{UTS / Evaluasi Tengah Semester : melakukan validasi hasil penilaian, evaluasi dan perbaikan proses pembelajaran berikutnya} \\
    \midrule
    1 & Sub-CPMK 1: \cpmka  & \idka & \krta & \mpa & \mpa & \mppa & \bpa\\
    2 & Sub-CPMK 2: \cpmkb  & \idkb & \krtb & \mpb & \mpb & \mppb & \bpb\\
    3 & Sub-CPMK 3: \cpmkc  & \idkc & \krtc & \mpc & \mpc & \mppc & \bpc\\
    4 & Sub-CPMK 4: \cpmkd  & \idkd & \krtd & \mpd & \mpd & \mppd & \bpd\\
    \midrule
    16 & \multicolumn{7}{c}{UAS / Evaluasi Akhir Semester : melakukan validasi  penilaian akhir, menentukan kelulusan mahasiswa} \\
    \bottomrule
\end{longtable}

\textbf{CATATAN :}
\begin{enumerate}
\item TM : Tatap Muka, BT: Belajar Terstruktur, BM: Belajar Mandiri
\item  (TM:2x(2x50')) dibaca : kuliah tatap muka 2 kali (minggu) x 2 SKS x 50 menit = 200 menit (3,33 jam)
\item  (BT+BM:(2+2)x2x60')) Dibaca : belajar terstrukur 2 kali (minggu) dan belajar mandiri 2 kali (minggu) x 2 SKS x 60 menit = 480 menit (8 jam).
\item  RPS: Rencana Pembelajaran Semester, RMK : Rumpun Mata Kuliah, PRODI: Program Studi.
\end{enumerate}



%----------------------------------------------------------------------------------------
%	BIBLIOGRAPHY
%----------------------------------------------------------------------------------------

\bibliographystyle{apalike}

\bibliography{biblio.bib}



\end{document}
